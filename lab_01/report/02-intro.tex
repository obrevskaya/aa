\chapter*{Введение}
\thispagestyle{empty}
\addcontentsline{toc}{chapter}{Введение}

Расстояние Левенштейна (редакционное расстояние) -- метрика, измеряющая разность между двумя последовательностями символов. Она определяется как минимальное количество односимвольных операций (а именно вставки, удаления, замены), необходимых для превращения одной последовательности символов в другую. Впервые задачу нахождения редакционного расстояния поставил в 1965 году советский математик Владимир Левенштейн при изучении последовательностей, состоящих из 0 и 1 \cite{Levenshtein}.

Расстояние Дамерау -- Левенштейна является модификацией расстояния Левенштейна: к операциям вставки, удаления и замены символов, определённых в расстоянии Левенштейна добавлена операция транспозиции (перестановки) символов.

Расстояние Левенштейна и его обобщения активно применяются:
\begin{itemize}
	\item для исправления ошибок в слове (в поисковых системах, базах данных, при вводе текста);
	\item в биоинформатике для сравнения генов, хромосом и белков.
\end{itemize}

Цель данной лабораторной работы -- получение навыков динамического программирования на примере задачи поиска редакционных расстояний.


\hyphenation{Да-ме-рау-Ле-венштейна}
Задачи данной лабораторной работы:
\begin{enumerate}[label=\arabic*)]
	\item изучение алгоритмов нахождения расстояния Левенштейна и\\ Дамерау -- Левенштейна;
	\item реализация алгоритмов Левенштейна и Дамерау — Левенштейна;
	\item выполнение замеров затрат реализаций алгоритмов по памяти;
	\item выполнение замеров затрат реализаций алгоритмов по процессорному времени;
	\item проведение сравнительного анализа двух нерекурсивных алгоритмов;
	\item проведение сравнительного анализа алгоритмов поиска расстояния Дамерау -- Левенштейна
\end{enumerate}