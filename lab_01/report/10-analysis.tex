\chapter{Аналитическая часть}

В данном разделе будут разобраны алгоритмы нахождения расстояния --
алгоритмы Левенштейна и Дамерау-Левенштейна.

Расстояние Левенштейна между двумя строками -- это минимальное количество операций, необходимых для превращения одной строки в другую.

Цены операций могут зависеть от вида операции (вставка, удаление, замена) и/или от участвующих в ней символов, отражая разную вероятность разных ошибок при вводе текста, и т. п. В общем случае:

\begin{itemize}
	\item $w(a,b)$ — цена замены символа $a$ на символ $b$.
	\item $w(\lambda,b)$ — цена вставки символа $b$.
	\item $w(a,\lambda)$ — цена удаления символа $a$.
\end{itemize}

Для решения задачи о редакционном расстоянии необходимо найти последовательность замен, минимизирующую суммарную цену. Расстояние Левенштейна является частным случаем этой задачи при

\begin{itemize}
	\item $w(a,a)=0$.
	\item $w(a,b)=1, \medspace a \neq b$.
	\item $w(\lambda,b)=1$.
	\item $w(a,\lambda)=1$.
\end{itemize}

Для расстояния Дамерау -- Левенштейна вводится редакторская операция --
транспозиция. Цена транспозиции двух соседних символов -- $w(ab,ba)=1$.

\section{Нерекурсивный алгоритм поиска расстояния Левенштейна}

Расстояние Левенштейна между двумя строками $S_1$ и $S_2$ может быть вычислено по реккурентной формуле \ref{eq:D}, где $|S_1|$ означает длину строки $S_1$, $S_1[i]$ — $i$-ый символ строки $S_1$, функция $D_{S_1, S_2}(i, j)$ определена как:

\begin{equation}
\label{eq:D}
D_{S_1, S_2}(i, j) = \begin{cases}
0, &\text{$i = 0$, $j = 0$}\\
i, &\text{$j = 0$, $i > 0$}\\
j, &\text{$i = 0$, $j > 0$}\\
\min \lbrace\\
\qquad D_{S_1, S_2}(i, j-1) + 1,\\
\qquad D_{S_1, S_2}(i-1, j) + 1, &\text{$i > 0, j > 0$}\\
\qquad D_{S_1, S_2}(i-1, j-1) + m(S_1[i], S_2[j])\\
\rbrace
\end{cases},
\end{equation}

а функция $m$ определена по формуле \ref{eq:m} как:
\begin{equation}
	\label{eq:m}
	m(a, b) = \begin{cases}
		0, &\text{если $a = b$}\\
		1, &\text{иначе}
	\end{cases}.
\end{equation}

Для оптимизации нахождения расстояния Левенштейна используют матрицу в целях хранения соответствующих промежуточных значений.
В таком случае алгоритм представляет собой построчное заполнение матрицы $M_{(|S_1|+1), (|S_2|+1)}$ значениями $D(i, j)$. В результате расстоянием Левенштейна будет ячейка матрицы с индексами $i = |S_1|$ и $j = |S_2|$.

\section{Нерекурсивный алгоритм поиска расстояния Дамерау -- Левенштейна}

Является модификацией расстояния Левенштейна -- добавлена операции транспозиции, то есть перестановки, двух символов.

Расстояние Дамерау — Левенштейна может быть найдено по формуле \ref{eq:d}, которая задана как


\begin{equation}
	\label{eq:d}
	d_{a,b}(i, j) = \begin{cases}
		\max(i, j), &\text{если }\min(i, j) = 0,\\
		\min \lbrace \\
			\qquad d_{a,b}(i, j-1) + 1,\\
			\qquad d_{a,b}(i-1, j) + 1,\\
			\qquad d_{a,b}(i-1, j-1) + m(a[i], b[j]), &\text{иначе}\\
			\qquad \left[ \begin{array}{cc}d_{a,b}(i-2, j-2) + 1, &\text{если }i,j > 1;\\
			\qquad &\text{}a[i] = b[j-1]; \\
			\qquad &\text{}b[j] = a[i-1]\\
			\qquad \infty, & \text{иначе}\end{array}\right.\\
		\rbrace
		\end{cases},
\end{equation}

Формула выводится по тем же соображениям, что и формула (\ref{eq:D}).

В результате расстоянием Дамерау -- Левенштейна будет ячейка матрицы с индексами $i = |S_1|$ и $j = |S_2|$.

\section{Рекурсивный алгоритм поиска расстояния Дамерау -- Левенштейна}

Рекурсивный алгоритм реализует формулу \ref{eq:d}.
Функция $D$ составлена из следующих соображений:
\begin{itemize}
	\item для перевода пустой строки в пустую требуется 0 операций;
	\item для перевода пустой строки в непустую строку $S_1$ требуется $|S_1|$ операций;
	\item для перевода непустой строки $S_1$ в пустую требуется $|S_1|$ операций;
\end{itemize}
Для перевода из строки $S_1$ в строку $S_2$ требуется выполнить последовательно некоторое количество операций (удаление, вставка, замена, транспозиция). Полагая, что $S_1', S_2'$ и $S_1'', S_2''$ строки $S_1$ и $S_2$ без одного и двух последних символов соответственно, цена преобразования из строки $S_1$ в строку $S_2$ может быть выражена как:
	\begin{enumerate}[label={\arabic*)}]
		\item сумма цены преобразования строки $S_1$ в $S_2'$  и цены проведения операции вставки, которая необходима для преобразования $S_2'$ в $S_2$;
		\item сумма цены преобразования строки $S_1'$ в $S_2$ и цены проведения операции удаления, которая необходима для преобразования $S_1$ в $S_1'$;
		\item сумма цены преобразования из $S_1'$ в $S_2'$ и операции замены, предполагая, что $S_1$ и $S_2$ оканчиваются на разные символы;
		\item цена преобразования из $S_1'$ в $S_2'$, предполагая, что $S_1$ и $S_2$ оканчиваются на один и тот же символ.
		\item сумма цены преобразования из $S_1''$ в $S_2''$, при условии, что два последних символа $S_1$ равны двум последним символам $S_2$ после транспозиции.
	\end{enumerate}
Минимальной ценой преобразования будет минимальное значение приведенных вариантов.

\section{Рекурсивный с кешированием алгоритм поиска расстояния Дамерау -- Левенштейна}

Рекурсивный алгоритм заполнения можно оптимизировать по времени
выполнения с использованием матричного алгоритма. Суть данного метода
заключается в заполнении матрицы при выполнении рекурсии. В случае, если рекурсивный алгоритм выполняет прогон для данных, которые еще не
были обработаны, результат нахождения расстояния заносится в матрицу. В
случае, если обработанные ранее данные встречаются снова, то есть ячейка
матрицы уже заполнена, для них расстояние не находится и алгоритм переходит к следующему шагу.

\clearpage
