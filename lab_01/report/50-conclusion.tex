\chapter*{Заключение}
\thispagestyle{empty}
\addcontentsline{toc}{chapter}{Заключение}

В ходе выполнения лабораторной работы поставленная цель была достигнута: были получены навыки динамического программирования на примере задачи поиска редакционных расстояний.

Были изучены и реализованы алгоритмы нахождения расстояния Левенштейна и Дамерау -- Левенштейна. Были выполнены замеры затрат реализаций алгоритмов по памяти и по процессорному времени, а также проведены сравнительный анализ двух нерекурсивных алгоритмов и сравнительный анализ алгоритмов поиска расстояния Дамерау -- Левенштейна.

В ходе экспериментов было продемонстрировано, что самым затратным по времени является рекурсивный алгоритм Дамерау -- Левенштейна, а по памяти -- рекурсивный с кешированием алгоритм Дамерау -- Левенштейна. Итеративные алгоритмы напротив являются наименее затратными по времени.